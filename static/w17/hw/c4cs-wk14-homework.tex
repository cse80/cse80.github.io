\documentclass{article}
\usepackage[T1]{fontenc}

\usepackage{amssymb}
\usepackage{courier}
\usepackage{fancyhdr}
\usepackage{fancyvrb}
\usepackage[top=.75in, bottom=.75in, left=.75in,right=.75in]{geometry}
\usepackage{graphicx}
\usepackage{lastpage}
\usepackage{listings}
\lstset{basicstyle=\small\ttfamily}
\usepackage{mdframed}
\usepackage{parskip}
\usepackage{soul}
\usepackage{tabularx}
\usepackage{textcomp}
\usepackage{upquote}
\usepackage{xcolor}

% http://www.monperrus.net/martin/copy-pastable-ascii-characters-with-pdftex-pdflatex
\lstset{
upquote=true,
columns=fullflexible,
keepspaces=true,
literate={*}{{\char42}}1
         {-}{{\char45}}1
         {^}{{\char94}}1
}
\lstset{
  moredelim=**[is][\color{blue}\bf\small\ttfamily]{@}{@},
}

% http://tex.stackexchange.com/questions/40863/parskip-inserts-extra-space-after-floats-and-listings
\lstset{aboveskip=6pt plus 2pt minus 2pt, belowskip=-4pt plus 2pt minus 2pt}



\usepackage[colorlinks,urlcolor={blue}]{hyperref}
\usepackage[capitalise,nameinlink,noabbrev]{cleveref}
\crefname{section}{Question}{Questions}
\Crefname{section}{Question}{Questions}

\begin{document}

\fancyhead[C]{\hl{Select the right page in Gradescope or we will not grade the question!}}
\fancyhead[L]{}
\fancyhead[R]{}

\fancyfoot[L]{\color{gray} C4CS -- W'17}
\fancyfoot[R]{\color{gray} Revision 1.0}
\fancyfoot[C]{\color{gray} \thepage~/~\pageref*{LastPage}}
\pagestyle{fancyplain}



\title{\textbf{Homework 14\\Profiling, Linting, and Static Analysis}}
\author{\textbf{\color{red}{Due: Saturday, April 15, 10:00PM (Hard Deadline)}}}
\date{}
\maketitle


\section*{Submission Instructions}
Submit this assignment on \href{https://gradescope.com/courses/5574}{Gradescope}.
You may find the free online tool \href{https://www.pdfescape.com}{PDFescape}
helpful to edit and fill out this PDF.
You may also print, handwrite, and scan this assignment.

\section*{Optional Reading}

\emph{Developer Survey 2017}, from Stack Overflow.

\url{https://stackoverflow.com/insights/survey/2017}

Some students have asked for some insights into CS in industry. Every year,
Stack Overflow does a large survey of developers. Their writeup caveats the
survey a bit, but it's worth calling out again that this is a self-selecting
group of the 50,000 people who are active or attentive enough to Stack
Overflow to see the survey and who are willing to spend the time to fill it
out. The response demographics are not necessarily representative of the
industry as a whole, nor should you feel obligated to fall into the bins that
the survey identifies. That said, I think it presents some interesting data.
They've been doing this for a few years now and looking over the history is
interesting as well.


\newpage


\section{Exploring Profiling\protect\footnote{%
  The \texttt{perf} profiling tools only work on Linux. You'll need a Linux
  environment (like your VM) for this assignment. If you primarily develop on
  a mac, XCode ships with a tool called \texttt{Instruments} that can be used
  to profile that may be worth exploring on your own.
}}

Clone a copy of
\url{https://gitlab.eecs.umich.edu/ppannuto/c4cs-f16-profiling}.

Look through \texttt{main.c} to understand what this program is doing (not
much, a ``nop'' is ``no-operation'', that is, sit idle for a cycle). The call
graph of this program is about:
{\small
\begin{verbatim}
\-- main
    \-- parent
            \-- child1
            \-- child2,child2,child2,...
    \-- no_children
\end{verbatim}
}

When run, this program will execute \texttt{16*LONG\_TIME} nop instructions.
By analyzing the code, what percentage of nop's should be attributed to each
function? Finish filling in the first column of this table:

\bigskip
\renewcommand{\arraystretch}{1.25} % space out the table rows
\textbf{
  \begin{tabular}{r r || r r r}
                  & \texttt{nop} Analysis & Profile 1 & Profile 2 & Profile 3 \\\hline
    main:         & 0\%                  & \rule{1cm}{0.15mm}\% & \rule{1cm}{0.15mm}\% & \rule{1cm}{0.15mm}\% \\
    parent:       & 6.25\%               & \rule{1cm}{0.15mm}\% & \rule{1cm}{0.15mm}\% & \rule{1cm}{0.15mm}\% \\
    child1:       & \rule{1cm}{0.15mm}\% & \rule{1cm}{0.15mm}\% & \rule{1cm}{0.15mm}\% & \rule{1cm}{0.15mm}\% \\
    child2:       & \rule{1cm}{0.15mm}\% & \rule{1cm}{0.15mm}\% & \rule{1cm}{0.15mm}\% & \rule{1cm}{0.15mm}\% \\
    no\_children: & \rule{1cm}{0.15mm}\% & \rule{1cm}{0.15mm}\% & \rule{1cm}{0.15mm}\% & \rule{1cm}{0.15mm}\% \\
  \end{tabular}
}

\bigskip
Now compile and profile the code 3 times, recording the percentage of time
attributed by \texttt{perf} to each function.

\textbf{Fill in the remaining three columns of the table with the results from
  these profiling runs.}

\textbf{Does the trend of time attributed to each function align with your
  expectations from the \texttt{nop} analysis?}
\begin{quote}
  \textbf{(Yes or No) \rule{2cm}{0.15mm}}
\end{quote}

\textbf{Give one reasonable explanation for why you think the measured
  percentages from profiling don't perfectly match the \texttt{nop} analysis:}
\vfill


\subsection{Call-Graph Profiling}
Try running \texttt{perf record -g ./main \&\& perf report -g}

The \texttt{-g} flag enables \texttt{call-graph} profiling. This causes to
profiler to pay attention to which functions called which. This can be a
slightly more intuitive profiling view, as the leftmost column is now the
percentage of time that this function and all its children ran. I find this
view makes finding the path of functions called to get to the time-consuming
function easier. In C programs, \texttt{main} will always sort to the top as
100\% of your code runs inside of \texttt{main}. As a thought exercise, what
code executes outside of \texttt{main} in C++?

In this example from my machine, we can see that the \texttt{parent} function
takes 65\% of the time, with 8\% of that coming from code \texttt{parent}
itself, and $42+15$\% coming from child functions:

{\footnotesize
\begin{verbatim}
+   99.99%     0.00%  main     main               [.] main
+   99.99%     0.00%  main     libc-2.23.so       [.] __libc_start_main
+   99.99%     0.00%  main     [unknown]          [.] 0x07be258d4c544155
+   64.51%     7.90%  main     main               [.] parent
+   42.00%    41.77%  main     main               [.] child2
+   34.86%    34.49%  main     main               [.] no_children
+   15.13%    15.02%  main     main               [.] child1
\end{verbatim}
}

\emph{\small
(There is no question you have to answer in section 1.1)
}

\newpage
\section{Trying out Static Analysis}

Recall from the debugging homework earlier this semester, we had some buggy
code that was trying to compute prime numbers.

\begin{verbatim}
git clone https://github.com/cse80/debugging-basics
git checkout gdb-debug-1       # The first bug
git checkout valgrind-debug    # The last bug
\end{verbatim}

Try running our static analysis tools on these two branches.

\textbf{%
  Does \texttt{cppcheck} find the first or last bug?
  Does \texttt{scan-build} find either bug?
}

\vspace{4em}


\subsection{On your own code!}

Run the static analysis tools on current or past code you've written. Recall
that by default, \texttt{cppcheck} does not enable all checks (check out the
lecture slide if you've forgotten).

\textbf{Copy one warning or error reported here:}\footnote{%
  In the unlikely case that no code you've ever written produces any warnings,
  write a small program that generates a message and use that here.
}
\vspace{4cm}

\textbf{Does this warning or error impact the correctness of your program? Why
  or why not?}
\vspace{3em}

\end{document}
